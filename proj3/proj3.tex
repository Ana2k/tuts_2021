\documentclass{article}
\usepackage[utf8]{inputenc}
\usepackage{xcolor}
\usepackage{graphicx}
\usepackage{hyperref}
\usepackage[margin=1.5cm]{geometry}
\hypersetup{colorlinks=true,linkcolor=blue,urlcolor=blue}

\title{Project 2 CS384 -   GUI Based Transcript Generator }
\author{Dr. Mayank Agarwal}
\date{Assignment Given: 31st Oct 2021,\\ Deadline 1st December 2021,  
23:59. Submission: GitHub }
\begin{document}
	\maketitle  
	\textbf{Things to be kept in mind} 
	\begin{enumerate}
%		\item Dont take any inputs from user . 
\item You need to make the project in a group of 2 students. If you wish you 
can do individually also.  

\item Ensure try catch is used so as to avoid issues while running the code. 

\item You can take online help, but you need to understand the code flow 
process.

\item Both partners need to commit the project to their respective github. 
\item Program will be checked for plagiarism.   
\end{enumerate}


You need to make the transcript for IIT Antartica after generating their 
results.  IIT Antartica since its far away, they were inspired from IIT Logo 
and replicated it. Refer the tut05 that you have already done, in which you
computed SPI CPI. Now 
the job is to make the SPI CPI output into a PDF for transcript (Check sample 
transcript: 0402CS99.pdf in 	transcriptsIITP folder. That needs to be 
replicated exactly). You can make a 
gui interface or a web interface. As always use python for processing  data but 
for pdf too you need to use a Python supported pdf generation library 
(pypdf..etc., there are many).

Options that needs to be provided. 

1. This 
interface will have option to enter a range of roll numbers. Then on single 
click the program should generate  all the transcript for that range. So if 
range is 0401cs10-0401cs15, it will generate 6 pdfs. In case the range is 
0401cs10-0401cs80 (and roll number for 0401cs60-0401cs80 dont exists, just dont 
generate the files for 0401cs60-0401cs80, but do generate the transcript whose 
roll numbers exists. Also, pop a list saying that these roll numbers dont 
exists.). A valid range 0401cs60-0401cs60, in this case only one transcript is 
generated 0401cs60. Save all file names in upper case. When provided a range, 
do case-insensitive matching. So  0401cs10-0401cs15,  0401cs10-0401CS15,  
0401CS10-0401cs15,  0401CS10-0401CS15 generates the same output. 

Generate all the 
transcripts in the folder:  transcriptsIITP

2. Option 2 is to generate all transcripts. Here given the grades.csv file, 
first extract all the unique roll numbers and generate all the transcripts in 
the folder:  transcriptsIITP

Filename will be \textbf{ROLL\_NUMBER.pdf} in UPPER CASE. 
If for a student only info till sem4 is available then transcript will be till 
sem4 only. Page size rule is same, btech a3, others a4.\\


\textbf{Page Size:}

Since BTech is 8 semesters so what happens here 
you can not fit all the marks on A4 paper so you need to make the PDF size to 
A3. However for MTech MSc and PhD the semester usually two or four so you can 
easily fit on A4 sheet so from the roll number check bit 3 and bit 4. That will 
provide you 
this information regarding btech, mtech, phd. 
01-btech, 11-mtech, 12-msc, 21 phd.
if Bit3/bit4 is 01 its btech (so a3 page) and if anything else 
then assume it to be masters/phd (so A4 page). 

Instead of Date of Issue line keep a line for Date Generated, and generate a 
system date dd-MM-yyyy hh:mm, e.g., 28 Oct 2021, 19:05\\
\textbf{Date Generated:}  28 Oct 2021, 19:05. Since this time is system 
generated, so 
it will have some offset when computing large number of transcript. So first 
transcript may have 
\textbf{Date Generated:}  28 Oct 2021, 19:05 and the last one may have 
\textbf{Date Generated:}  28 Oct 2021, 19:12 depending how many are being 
generated.

For signature and seal, there should be space provided. In the GUI, keep an 
option to upload SEAL and signature. Both shall be optional, if not provided, 
then it will be blank however, the text for signature (Assitant Registrar 
(Academic)) shall be there irrespctive if sign is there or not. 


Helpful links 
\href{https://towardsdatascience.com/creating-pdf-files-with-python-ad3ccadfae0f}{https://towardsdatascience.com/creating-pdf-files-with-python-ad3ccadfae0f}
 

\href{https://stackoverflow.com/questions/2252726/how-to-create-pdf-files-in-python}{https://stackoverflow.com/questions/2252726/how-to-create-pdf-files-in-python}


Check sample transcript: 0402CS99.pdf in 	transcriptsIITP folder.\\
IITP logo and hindi text is your responsibility to search and add.




%Sample Input and Output are provided
%  7 generation you need to generate transcript semester wise.  So for example 
%  if the information Up 2 sem 7 is available you need to also generate a total 
%  of seven transcript for that particular roll number write roll number\_sem 1 
%  roll number and usko sem 2 and so on up to roll number\_sem 7

%
%\textbf{Note: Please convert the roll number to upper case while saving the 
%files..}\\
%
%So correct answer will be marked with the roll number as ANSWER (in uppercase) 
%which can be in any row in the the csv (Column is fixed as said above)
%Create a python GUI mark sheet generator containing following key points:
%\begin{itemize}
%    \item Add a button where you can browse .csv file and check if ANSWER is 
%present or pop up error.
%    \item Enter the 5 marks for correct answers and -1 for wrong answer and 
%make a button \textbf{Compute Marksheet}. 
%    \item When you click on \textbf{Compute Marksheet} button,
%Note: The total marks are displayed after being calculated automatically.
%
%\item Now add a button \textbf{Generate Marks}.
%\end{itemize}

\end{document}
