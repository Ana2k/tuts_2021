\documentclass{article}
\usepackage[utf8]{inputenc}
\usepackage{xcolor}
\usepackage{graphicx}
\usepackage{hyperref}
\usepackage[margin=1.5cm]{geometry}
\hypersetup{colorlinks=true,linkcolor=blue,urlcolor=blue}

\title{Project 2 CS384 -  Class Grading Generator}
\author{Dr. Mayank Agarwal}
\date{Assignment Given: 17th Oct 2021,\\ Deadline 1st December 2021,  
23:59\\Submission: GitHub }
\begin{document}
	\maketitle  
	\textbf{Things to be kept in mind} 
	\begin{enumerate}
%		\item Dont take any inputs from user . 
\item You need to make the project in a group of 2 students. If you wish you 
can do individually also.  
\item Program will be checked for plagiarism.   
\end{enumerate}

As per IITPatna senate norms, these are the grading percentages that need to be 
given to a class. 

Grade Approx. \% of  students\\
AA 5\%  \\
AB 15\%  \\
BB 25\%  \\
BC 30\%  \\
CC 15\%  \\
CD 5\%  \\
DD \&   F 5\%   \\

So in a class of 100 students, the grade distribution will be as follows: 
AA 5,
AB 15,
BB 25,
BC 30,
CC 15,
CD 5,
DD 5.  

You need to make a web based program that takes a csv file as input, 
The first three columns are freezed 

\noindent \textbf{Sl No}: Serial Number \\
	\textbf{Roll No}: Roll number of the students\\
		\textbf{Name}: Name of the student\\ 

Rest of the columns depend on the number of assignments taken by the professor. 
If someone takes k assignments, then there will be k columns. 
Row 2 has \textbf{Max Marks} that contains the max marks of a particular 
assignment. Row 3 has the \textbf{Weightage} that is the weightage that is 
assigned to every assignment. 

Your program should check that the total weightage should be equal to 100 
first, then compute the grades. So each of the k assignment is first converted 
into a scale of 100 marks then scaled to the respective weightage assigned. 



 

\end{document}
