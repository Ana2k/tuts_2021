\documentclass[12pt,  letterpaper,  twoside]{article}
\usepackage[utf8]{inputenc}
\usepackage{xcolor}
\usepackage{hyperref}
\usepackage[margin=1.5cm]{geometry}
\hypersetup{colorlinks=true,linkcolor=blue,urlcolor=blue}

\title{Tutorial 06 - CS384 -  Web Series Renamer \& Regular Expression}
\author{Dr. Mayank Agarwal}
\date{Assignment Given: 17th Oct 2021,\\ Deadline 23rd Oct 2021,  
	23:59\\Submission: GitHub }

\begin{document}
	\maketitle  
	\textbf{Things to be kept in mind}\\
	\begin{enumerate}
		%		\item Dont take any inputs from user .  
		\item Program will be checked for plagiarism.   
		\item Using Regex is mandatory
		
	\end{enumerate}
	
	A sample set of \textbf{srt+mp4} is given to you in separate folders. You 
	need to rename \textbf{mp4 + srt files}. You know that in VLC/any media 
	play will auto include the srt if the filename and srt name is same. You 
	need to ask 3 inputs from the users
	
	\begin{enumerate}
		\item Initially make a menu based program that will ask which web 
		series to 
		rename. The sample python code I have already given along with the 
		variables.
		\item Season Number Padding (take int input): The padding given to the 
		season number.  A padding of three digits means if the series have 20 
		seasons then it will be renamed as season 001, 002, 003..., 019, 020.
		\item Episode Number Padding (take int input): For example,  a padding 
		of two means if the series have 30 episodes then it will be renamed as 
		episode 001, 002, ..., 030
	\end{enumerate}
	
	There are 3 folders that you need to rename:
	\begin{enumerate}

\item  Breaking Bad
\item   Game of Thrones
\item 	Lucifer

	\end{enumerate}
	
	Lets say the user provides these inputs: 

	
	\begin{enumerate}
		\item Main Title of the \textbf{Web Series}: 2 
		\item Season Number Padding (take int input): 1
		\item Episode Number Padding (take int input): 2
	\end{enumerate} 
	
	
	
	\noindent
	\textbf{Original}:: `Game of Thrones - 8x05 - The 
	Bells.WEB.REPACK.MEMENTO.en.srt"  \\
	\textbf{New}: ``Game of Thrones - Season 8 Episode 05 - 
	The 
	Bells.mp4" \\
	
	Also look,  there is an interesting case:
	Look at one of the names:
	
	``Game of Thrones - 8x05 - The 
	Bells.WEB.REPACK.MEMENTO.en.srt''
	 
	
	
	\noindent
\textbf{Original}:: `Game of Thrones - 8x05 - The 
Bells.WEB.REPACK.MEMENTO.en.srt"  \\
\textbf{New}: ``Game of Thrones - Season 8 Episode 05 - 
The 
Bells.mp4" \\
	
	Here you can see that ".WEB.REPACK.MEMENTO.en" is the string that is not 
	required. 
	Usually every srts have common strings towards the end that needs to be 
	identified manually and stripped. So in case of Game of Thrones,  you need 
	to find 
	out manually such non-useful string and strip them in the final code. A 
	\textbf{recommended} approach will be \textbf{regex}. Like 
	\textbf{splitting} on the \textbf{WEB} and other such string (e.g., 
	\textbf{720p}).
	
	\textbf{Task:} Based on the received input,  the filename should be 
	\textit{seriesname season number (depending on the padding) episode number 
	(depening on the padding) + episode name ()if any)}.  
	
	
	Following series needs to be renamed (srt+mp4) via the os module. 
	
	\begin{enumerate}
		\item Breaking Bad
		\item Game of Thrones
				\item Lucifer
	\end{enumerate}
	
	\textbf{
		Spaces should be strictly one between characters and numbers.}
	
	
\end{document}