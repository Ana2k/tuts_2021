\documentclass[12pt]{article}
\usepackage{graphicx}
\title{CS384 2020 Assignment 5 - Web Series Renamer}
\author{Mayank Agarwal}
%\date{14 Oct 2020}
\usepackage[margin=1cm]{geometry}
\usepackage{xcolor}
\begin{document}
{\let\newpage\relax\maketitle} 

\textbf{Git commits, one for each web series, there are 5 web series, so 5 commits atleast}


\textbf{Assignment Given on 4th Nov 2020, 2100 hrs}

\textbf{Assignment Deadline on 9th Nov 2020, 2359 hrs}\\

A sample set of \textbf{srt+mp4} is given to you in separate folders. You need to rename \textbf{mp4 + srt files}. You know that in VLC/any media play will auto include the srt if the filename and srt name is same. You need to ask 3 inputs from the users

\begin{enumerate}
	\item Main Title of the \textbf{Web Series}: This will contain the series name. E.g.,  \textbf{Game of Thrones},  \textbf{Suits} etc. 
	\item Season Number Padding (take int input): The padding given to the season number.  A padding of three digits means if the series have 20 seasons then it will be renamed as season 001, 002, 003..., 019, 020.
	\item Episode Number Padding (take int input): For example,  a padding of two means if the series have 30 episodes then it will be renamed as episode 001, 002, ..., 030
\end{enumerate}


Check for example one the srt and mp4 in the ``How I Met Your Mother" folder. 
Original name:  ``How I Met Your Mother - 1x07 - Matchmaker.en.srt" 
Lets say the user provides these inputs: 


\begin{enumerate}
	\item Main Title of the \textbf{Web Series}: How I Met Your Mother
	\item Season Number Padding (take int input): 1
	\item Episode Number Padding (take int input): 2
\end{enumerate} 



\noindent
\textbf{Original}:: ``How I Met Your Mother - Season 1 Episode 07 - Matchmaker.srt"  \\
\textbf{New}: ``How I Met Your Mother - Season 1 Episode 07 - Matchmaker.mp4" \\

For Suits,  there is an interesting case:
Look at one of the names:

``Suits - 3x03 - Unfinished Business.HDTV.EVOLVE.en.srt''

Suppose user provides the following: 
\begin{enumerate}
	\item Main Title of the \textbf{Web Series}: Suits
	\item Season Number Padding (take int input): 2
	\item Episode Number Padding: 2
\end{enumerate}


\noindent
\textbf{Original}:
``Suits - 3x03 - Unfinished Business.HDTV.EVOLVE.en.srt'' \\
\textbf{New}:
``Suits - Season 03 Episode 03 - Unfinished Business.srt''\\

Here you can see that ".HDTV.EVOLVE.en" is the string that is not required. Usually every srts have common strings towards the end that needs to be identified manually and stripped. So in case of Suits,  you need to find out manually such non-useful string and strip them in the final code. A \textbf{recommended} approach will be \textbf{regex}. Like \textbf{splitting} on the \textbf{HDTV} and other such string (e.g., \textbf{720p}).

\textbf{Task:} Based on the received input,  the filename should be \textit{seriesname season number (depending on the padding) episode number (depening on the padding) + episode name ()if any)}.  


Following series needs to be renamed (srt+mp4) via the os module. 

\begin{enumerate}
	\item FIR (I used to watch it a lot and still a fan)
	\item Suits (Credits: Kaushik Komanduri )
	\item How I Met Your Mother  (Credits: Kaushik Komanduri)
\item 	Sherlock (Credits: Ashutosh Anand)
\item Game of Thrones
\end{enumerate}

\textbf{
	Spaces should be strictly one between characters and numbers.}

\end{document}
