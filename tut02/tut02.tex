\documentclass[12pt,  letterpaper,  twoside]{article}
\usepackage[utf8]{inputenc}
\usepackage{xcolor}
\usepackage{hyperref}
\usepackage[margin=1.5cm]{geometry}
\hypersetup{colorlinks=true,linkcolor=blue,urlcolor=blue}

\title{Tutorial 02 CS384 -  Memory Game Coding}
\author{Dr. Mayank Agarwal}
\date{Assignment Given: 20th Aug 2021,\\ Deadline 23th August 2021,  
23:59\\Submission: GitHub }

\begin{document}
	\maketitle  
	\textbf{Things to be kept in mind}\\
	\begin{enumerate}
%		\item Dont take any inputs from user . 
		\item You should first check that the list does not contain any 
		floating numbers  or characters. The list should have only integers as 
		input. If 
		a non int is present, just output: ``Please enter a valid input list" 
		and show all the elements that was invalid. E.g, 
		if input\_nums = [1,4, "a", 3, 5, "Star", 7.5], then output will be:
		``Please enter a valid input list". Invalid inputs detected": [``a", 
		``Star", 7.5] and exit the code. You can use  
		isdigit() 
		function also. 
		
		\item While evaluating your program, the TA we will modify the input 
		list (input\_nums), and 
		check for 
		correctness. End user will not enter any input via keyboard, the input 
		will be 
		static in nature, input\_nums = []
		\item Program will be checked for plagiarism.  
	\end{enumerate}
	
	
	\# Returns the score from a memory game.  The strategy is to remove the 
	number that has
	 been in the memory the longest time.  
	
	A memory game is played (and scored) as follows: Random numbers between 0 
	and 10 (zero inclusive) are called out one at a time. In this memory game 
	the player can remember a maximum of 5 previously called out numbers. If 
	the called number is already in the player's memory, a point is added to 
	the player's score. If the called number is not in the player's memory, the 
	player adds the called number to his memory, first removing another number 
	if his memory is full. In our simulation of this game, the number which is 
	removed from the player's memory is the number that has been in the 
	player's memory the longest time. For example, if the random numbers are 
	[3, 4, 3, 0, 7, 4, 5, 2, 1, 3], the game proceeds as follows:
	
	\noindent Called number 3: Score: 0, Numbers in memory: [3]\\
	Called number 4: Score: 0, Numbers in memory: [3, 4]\\
	Called number 3: Score: 1, Numbers in memory: [3, 4]\\
	Called number 0: Score: 1, Numbers in memory: [3, 4, 0]\\
	Called number 7: Score: 1, Numbers in memory: [3, 4, 0, 7]\\
	Called number 4: Score: 2, Numbers in memory: [3, 4, 0, 7]\\
	Called number 5: Score: 2, Numbers in memory: [3, 4, 0, 7, 5]\\
	Called number 2: Score: 2, Numbers in memory: [4, 0, 7, 5, 2]\\
	Called number 1: Score: 2, Numbers in memory: [0, 7, 5, 2, 1]\\
	Called number 3: Score: 2, Numbers in memory: [7, 5, 2, 1, 3]\\
	
	Complete the get\_memory\_score() function which is passed a list of random 
	numbers as a parameter and returns the final score using the algorithm 
	described above. For example, the following code:\\
	print("1. Score:", get\_memory\_score([3, 4, 1, 6, 3, 3, 9, 0, 0, 0]))\\
	print("2. Score:", get\_memory\_score([1, 2, 2, 2, 2, 3, 1, 1, 8, 2]))\\
	print("3. Score:", get\_memory\_score([2, 2, 2, 2, 2, 2, 2, 2, 2]))\\
	print("4. Score:", get\_memory\_score([1, 2, 3, 4, 5, 6, 7, 8, 9]))\\
	input\_nums = [7, 5, 8, 6, 3, 5, 9, 7, 9, 7, 5, 6, 4, 1, 7, 4, 6, 5, 8, 
	9, 
	4, 8, 3, 0, 3]\\
	print("5. Score:", get\_memory\_score(input\_nums))\\
	
	prints:\\
	1. Score: 4\\
	2. Score: 6\\
	3. Score: 8\\
	4. Score: 0\\
	5. Score: 10\\
	
	\textbf{Sample input output }\\
	
	\textbf{Input}\\
	\noindent	input\_nums = [3, 4, 1, 6, 3, 3, 9, 0, 0, 0]
	
	\textbf{Output}\\
	\noindent	
	Score: 4 \\ 
	
	
\end{document}